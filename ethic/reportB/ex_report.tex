\documentclass[11pt, a4paper]{article}
\usepackage{graphicx, fullpage, hyperref, listings}
\usepackage{appendix, pdfpages, color}

\usepackage{tocloft}            % This squashes the Table of Contents a bit
\setlength\cftbeforesecskip{3pt}

\definecolor{MyLightYellow}{cmyk}{0,0.,0.2,0}

\setlength{\parskip}{4pt}        % sets spacing between paragraphs
\interfootnotelinepenalty=500    % this prevents footnotes breaking across pages

\title{\includegraphics[width=0.4\textwidth]{UnivCrest}
        \\Ethics B}          % <<<<<<<<< change the title as appropriate
\author{%\textcolor{red}{Student ID 200123456 (Group 999)}                % <<<<<<<<< your ID and group number
        \\  \textcolor{red}{ELEC273}\footnote{\textcolor{red}{IMPORTANT: In a standard technical report, you would need to include here your personal details as the author of the document. However, remember that marking of coursework is anonymous and therefore you should remove this part before submitting your report for Year 2 labs! Do not include your name, student ID, email address or any other personal information.}}}                                    % <<<<<<<<< module code
\date{\tiny{\today}}

\begin{document}
\begin{titlepage}
\maketitle
\addtocontents{toc}{\protect\thispagestyle{empty}} % because we don't want a page number on the title page
                                                   % Thanks to Huang Shanyue for suggesting this


\fbox{
\begin{minipage}{0.9\linewidth} \footnotesize
\begin{center} \textbf{Declaration} \end{center}
I confirm that I have read and understood the University's definitions of plagiarism and collusion from the Code of Practice on Assessment. I confirm that I have neither committed plagiarism in the completion of this work nor have I colluded with ant other party in the preparation and production of this work. The work presented here is my own and in my own words except where I have clearly indicated and acknowledged that I have quoted or used figures from published or unpublished sources (including the web). I understand the consequences of engaging in plagiarism and collusion as described in the Code of Practice on Assessment (Appendix L).
\end{minipage}
}

\thispagestyle{empty}
\tableofcontents
\end{titlepage}


%-------------------------------------------------------------------------------------------------------
\section{Scenario}
%-------------------------------------------------------------------------------------------------------
You are the member of a team responsible for the regular maintenance of a fleet of 10 aircraft for a small airline. Recently, a jet has caught fire and crashed over the North Sea and your team is charged with examining the wreckage to see if any malfunction occurred. They quickly identify that the crash was due to the failure of a pump casing; in particular, studs that attach the pump casing had failed. You put these findings into a report which you pass on to your superior.

Three days later a memo is circulated by Head Office instructing all maintenance teams to replace all pump casing studs on every jet of this kind in the fleet. Replacement studs are delivered to the workshops with instructions that this job is to be undertaken as a matter of urgency, with crews working over the weekend to ensure its swift completion.

You examine the new studs that are to be put in all aircraft. You come to the conclusion that they are of poor quality: the studs have cut threads where the old studs had rolled threads (rolled threads have a better fatigue resistance). You are not convinced that the replacement studs have the physical capacity to keep the pump casings on securely and believe that this may lead to further accidents.

However, the order to replace the studs has come from ‘on high’, and under normal circumstances you would be under an obligation to obey such an order. You are hoping for promotion in the next 6 months and do not want to be marked out as a troublemaker.


\section{Dilemma}

You work on the regular maintenance of 10 airplanes for a small airline. You are given orders to replace all the pump casing studs on the aircraft swiftly. If you can finished this task as ordered, you are expected to promote in the next 6 months. However, as you had participated the investigation of a recent wreckage, you believe that the orders from the superiors can cause more risk of aircraft crash as it applied a poorer quality of studs. You want to just follow the orders and wait for promotion but you also want to report this information to your superior because of your honesty, integrity and respect for life.

\section{What Could Be Done?}

\begin{enumerate}
\item Just follow the orders from the superior.
\item Inform the superior your worries concerning the risk of studs replacement and provide him the relevant information you have. If your superior is not willing to accept you for no reason, try to reflect this to the boss of your company directly.
\item You can find media for help so that your company will pay special attention to the risk and protect the life and public good.
\end{enumerate}




\section{Discussion}

For the first option, which is just follow the order from the superior, there are mainly three reasons for doing this. Firstly, my conclusion that changing stubs can increase the risk of crash might be wrong. As in the investigation, all information I found had been reported to the superior, and the superior finally decided to change all the problem stubs on all aircraft. I had done my work perfectly and I don't known whether the superior have other consideration. Furthermore, if I work just following the orders and perform like an obedient employee, I can get promoted soon.

However, if I choose the first option, more worse things can be caused. Firstly, as the final report of the crash was submitted by me, it can be suggested that perhaps I am the most knowledgeable person in this accident and I have duty to make it clear why the superior decides to change the stubs to a poor-quality one. Is this because that I did not explain the accident reason clearly in the accident report? Or is this because that the superior did not have enough professional knowledge on this? This must be figured out because that if this order really done by mistake, customers' life and good might be harmed and I would be suspected since the last similar crash investigation was leaded by me and this time there is another crash for the same reason. From this it can be indicated that just follow the orders do not always good for promotion, instead, it could destroy my career.

If I choose the second option, which is to inform the superior your concerns, I will feel comfortable in my heart because I do not obey honest and integrity. Because of your action, many people's life are saved and your responsibility for investigating the last aircraft crash is achieved. Besides, if you warning your superior properly and quickly, and pointed out the big risk of the next crash with your professional explanation, it seems that your superior will not regard you as a trouble maker, but a responsible engineer as the wrong decision was made by him, not you. Your action can also save a great amount of money as there might be some stubs that not being delivered.

Nevertheless, it is also possible that the superior not willing to accept your suggestion. Except that he or her refused you because your worry comes from your lack of professional knowledge, and the new stubs will not improve the risk of crash, but decrease it. Otherwise, it is suggested that the top leaders such as the boss might need to be informed to know the details as these kind of leaders always more care about the company image from the public instead of the little extra cost of changing stubs. Besides, other uncertainty also exist such as that the feedback management in your company is poor and the process of reporting your concerns is very complex and may take a long time. Apart from these, your if you decide to choose the option two, your career will become more uncertain no matter the result is good or bad.

If you are tired with your company or you find that the second option is not effective and you want to fulfill your responsibility, the third option that communicating with media is suggested. The advantage of the option is that after the exposure of this issues in the airline industry, public and government may urge the industry to take action to prefect the management thus reduce the occurrence of similar accident. 

Notwithstanding, this external measurement will certainly harm your company's reputation and profit. And if your company is small, this behavior can actually destroy your company. Even if your company can survive from this public relations crisis, you will certainly lose the trust of your company and most of your colleagues. Inevitably, when you want to find a new engineer job from other companies, these companies are more likely to refuse you as they may think that you will not respect them because of what you have done on your last employer.

\section{Recommendation}

From the discussion section, it is suggested that the option two, which is to report your concerns to your superiors and solve this problem in your company is the most suggested options. Although this might give rise to more uncertain to your career, this option will help you fulfill your social responsibility and let you feel comfortable in your heart. Deferent from option 2, option 1 and 3 will harm your reputation to different extent and in different dimensions. If you follow option 1, when there is another air crash because of the same reason as the former one, you as the leader of the last investigation will certainly be suspected. As for option 3, if you want to try the media method, even if this way seems to be the most effective one from the perspective of avoid accident, you will create enemies for yourself in your company and industry. And this environment will effectively affect your career.

As for the principles used in discussion and the mode of thinking, the item of `respect life, law and the public good` is the most important one in this case. The reason that I decided to not just follow the order from the superior is that I want to save live and protect public good. If I do not god so, I may feel uncomfortable. The second principle is the `honesty and integrity`. This point is performed in the action that I respect the reputation of my employer. If I can figure out a way to correct the company's action effectively, the method of utilizing media will not be considered. Thirdly, the principle of `accuracy and rigour` can be indicated from the double check of your investigation data and conclusion.
 



% --------------------------- This is how to declare the Appendices section ----------------------------
\newpage


\end{document}
